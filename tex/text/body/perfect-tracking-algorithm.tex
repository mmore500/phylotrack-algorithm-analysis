\section{Perfect Tracking Algorithm} \label{sec:perfect-tracking-algorithm}

Traditionally, the problem of recording the phylogenies of replicating digital populations has been solved using a ``perfect tracking'' approach where all replication events are recorded and stored in a tree (or forest, if there are multiple roots) \citep{dolson2023phylotrackpy}.
In this section, we present results on the time and space complexity of perfect tracking;
these results will serve as a basis of comparison for hereditary stratigraphy's curation policy and reconstruction algorithms in Section \ref{sec:discussion}.

Perfect tracking algorithms are designed to augment an ongoing replication process by performing supplementary record-keeping.
As such, perfect tracking algorithms execute three separate routines: 1) initialization, 2) handling the birth of new individuals, and 3) (optionally) handling the removal of existing individuals.
The initialization step happens once before the founding set of population members is introduced.
The birth handling routine executes any time an individual is replicated.
Similarly, the removal-handling routine gets called any time an individual is removed from the set of currently active individuals (i.e., the population).

Perfect tracking can be used regardless of whether each individual has one or multiple parents (i.e., asexual vs. sexual reproduction).
Our analysis here will focus on the asexual scenario where each individual has only parent, but most of our results generalize to the multi-parent case.
The primary exception is the results presented in Section \ref{sec:perfect-tracking-pruning-space}, which will underestimate memory use in the average multi-parent case.

Perfect tracking can be done at an arbitrary level of abstraction \citep{dolson2020interpreting}.%, dolson2023phylotrackpy}.
For example, individual population members could be grouped into more abstract taxonomic units on the basis of a particular genetic or phenotypic trait.
As in biological phylogenetics, ancestry trees can show relationships between individuals, species, or any other taxonomic unit.
Increasing the level of abstraction decreases memory use by a constant (but often large) factor.
Here, we will assume that the entities being tracked are the underlying replicating individuals, as this case is the most computationally expensive (i.e., worst case).

In the following subsections, we present two different algorithms for perfect phylogenetic tracking: one which naively accumulates records for every individual that ever existed and another which reclaims memory as lineages go extinct.

\subsection{Naive perfect tracking}
\label{sec:naive-perfect-tracking}

The naive perfect tracking algorithm is fairly straightforward.
It is formally desribed in Algorithm \ref{alg:perfect-tracking}.

\renewcommand{\algorithmiccomment}[1]{ \newline \textbf{comment:} \textit{#1}}
\renewcommand{\algorithmicrequire}{\textbf{Input:}}

\begin{algorithm}
    \caption{Perfect phylogenetic tracking}
    \label{alg:perfect-tracking}
    \textbf{Part 1: Initialization}
    \begin{algorithmic}[1]
    \State{Define $T:=$ an empty forest}
    \end{algorithmic}
    \textbf{Part 2: Handle birth event}
    \begin{algorithmic}[1]
        \Require{$P_i$ = newly created organism, $P_j$ = $P_i$'s parent}
        \State{Add node representing $P_i$ to $T$}
        \State{Add edge in $T$ from $P_j$ to $P_i$}
    \end{algorithmic}
\end{algorithm}


\subsubsection{Time complexity}

The naive perfect tracking algorithm can be implemented in constant time (see Theorem \ref{thm:perfect-tracking-time}).

% @ELD Could make this theorem better
\begin{theorem}{Naive Perfect Tracking Time Complexity}
\label{thm:perfect-tracking-time}
The naive perfect tracking algorithm can be implemented in constant time ($\mathcal{O}(1)$) per birth event.
\end{theorem}

\begin{proof}
\label{prf:perfect-tracking-time}
For the purposes of this proof, we use the RAM model of computation. 
Initialization takes a single simple operation and only happens once. 
Thus, it is trivially constant-time.
Handling a birth event takes two simple operations.
Thus, each run of the birth-handling algorithim trivially takes constant time.
The birth-handling algorithm will run once per birth event.
\end{proof}



Note that, although the number of birth events is likely not constant with respect to the size of the population or the number of generations,
these birth events are not part of the tracking algorithm and would have happened regardless of whether phylogeny tracking was in place.
Thus, we need only consider time complexity per birth event.

\subsubsection{Performance in parallel and distrbuted environments}
% @ELD Could do a formal analysis with the PRAM model here
\label{sec:perfect-tracking-distrbuted}

Naive perfect tracking can be adapted to parallel computing environments using universally unique agent identifiers through either of two straightforward options.
In the first approach, records of all agent creation and elimination events are collected to a centralized database as they occur.
In the second, each processor maintains an independent data file of local agent creation/elimination that are stitched together in postprocessing.
For the former, race conditions might occur where the phylogeny tracker is notified of a birth event involving a parent it does not yet know about.
This situation can be mitigated fairly cheaply by having a pool of ``waiting'' nodes that will be added once the appropriate parent node is added.

For both approaches to naive perfect tracking, bigger problems occur in distributed environments, where passing messages to a centralized data structure could become expensive \citep{moreno2022hereditary} or limited storage space is available to individual processor elements.
Worse still for either approach to naive perfect tracking is the possibility of data loss, which can become a significant likelihood at scale \citep{cappello2014toward}.
It is unclear how a perfect tracking algorithm could recover from missing data.

\subsubsection{Space complexity}

Naive perfect tracking has a gargantuan space complexity, owing to the fact that it maintains a tree containing a node for every object that ever existed (see Theorem \ref{thm:perfect-tracking-space}).

\section{Additional notes on the memory requirements of perfect tracking} \label{sec:perfect-tracking-space-supp}
% TODO: section name that isn't monstorously long

Note that perfect phylogeny tracking algorithms are designed to plug into a larger computational process.
For phylogeny tracking to be relevant, this computational process must have a collection of $N$ objects that are currently eligilble to be copied (in evolutionary terms, the population).
Therefore, the space complexity of this process must be at least $\mathcal{O}(N)$.
As long as the expected size of $T$ scales no faster than $\mathcal{O}(N)$, then, phylogeny tracking will not be the primary factor determining memory usage.
While the memory cost of phylogeny tracking could still be significant in these cases, it is unlikely to be the primary factor determining whether running the program is tractable.

Previously, we noted that cases where perfect tracking with pruning achieves its worst case size order of growth are esoteric.
There are two ways that theses situations can occur:

\begin{itemize}
\item $N$ is continuously increasing.
\item There is effectively no coalescence.
\end{itemize}

In case 1, the space complexity of the external computational process will also be growing at least as fast as $\mathcal{O}(N)$.
Because $N$ itself is growing, this behavior should overwhelm the size of the history in $T$ in most cases.
Even in the (completely unrealistic) case where $N$ is allowed to reach infinity, there is an argument to be made that the internal nodes of $T$ would not be substantially contributing to memory complexity.
While this argument is beyond the scope of this paper, it would center around the fact that most evolutionarily-realistic coalescence proceses ``come down from infinity'' \citep{berestyckiRecentProgressCoalescent2009}.
% Indeed, unless $N$ starts out infinite, any process observed by phylogenetic tracking must come down from infinity, as they 

Case 2 is a legitimate circumstance where worst case memory complexity could occur. 
It would occur in cases where every member of the population has (approximately) one offspring.
While this level of regularity is unlikely in the context of evolution, it could happen when tracking non-evolutionry digital artifacts.

\subsection{Pruning-enabled perfect tracking}
\label{sec:naive-perfect-tracking-with-pruning}

For most substantive applications, the amount of memory used by naive perfect tracking is prohibitive.
As such, it is common to use a ``pruning''-enabled algorithm that deletes records associated with extinct lineages (see Algorithm \ref{alg:perfect-tracking-pruning}). % TODO CITE \citep{dolson2023phylotrackpy}
This procedure removes all parts of the phylogeny that are no longer ancestors of currently extant objects.
Thus, like naive perfect tracking, it introduces no uncertainty into down-stream analyses.

\begin{algorithm}
    \caption{Perfect phylogenetic tracking with pruning}
    \label{alg:perfect-tracking-pruning}
    \textbf{Part 1: Initialization} \\
    Same as algorithm \ref{alg:perfect-tracking} \\
    \textbf{Part 2: Handle birth event}
    \begin{algorithmic}[1]
        \Require{$P_i$ = newly created organism, $P_j$ = $P_i$'s parent}
        \State{Add node representing $P_i$ to $T$}
        \State{Add edge in $T$ from $P_i$ to $P_j$}
        \State{Record that $P_i$ is alive}
        \State{Increment $P_j$'s count of living offspring lineages}
    \end{algorithmic}
    \textbf{Part 3: Handle removal event}
    \begin{algorithmic}[1]
        \Require{$P_i$ = newly removed organism}
        \State{$P_j$ $\gets$ parent of $P_i$}
        \State{Record that $P_i$ is not alive}

        \While{$P_i$ is not alive \textbf{and} $P_i$'s count of living offspring lineages == 0}
            \State{Remove $P_i$ from $T$}
            \State{Decrement $P_j$'s count of living offspring lineages}
            \State{$P_i$ $\gets$ $P_j$}
            \State{$P_j$ $\gets$ parent of $P_j$}
        \EndWhile

    \end{algorithmic}
\end{algorithm}


\subsubsection{Time Complexity}

The pruning-enabled algorithm has amortized constant time complexity (see Theorem \ref{thm:perfect-tracking-with-pruning-time}).

\begin{theorem}{Pruning Time Complexity}
\label{thm:perfect-tracking-with-pruning-time}
The time complexity of pruning in perfect phylogenetic tracking is $\mathcal{O}(1)$, amortized. 
\end{theorem}

\begin{proof}
\label{prf:perfect-tracking-with-pruning-time}
The initialization and birth-handling steps of the pruning algorithm are trivially constant-time, as explained in proof \ref{prf:perfect-tracking-time} (technically the pruning algorithm adds two constant-time operations to the birth-handling step, but this fact is inconsequential).
Next, we prove that the removal-handling pruning algorithm runs in amortized constant time.
When an arbitrary taxon $t$ is removed, there are three possibilities: 1) $t$ has living descendants, 2) $t$ has no living descendants, but $t$'s parent is either alive or has other descendants that are alive, or 3) $t$ was the last living descendant of its parent.
In case 1, nothing can be removed from the phylogeny.
Case 1 takes contant time, because it only requires executing a single comparison operation.
In case 2, only $t$ can be removed from the phylogeny.
Case 2 also takes constant time, as it only requires removing a single node from the tree and doing three comparison operations.

In case 3, pruning must be done.
We recurse back up the lineage until we find a taxon, $a$, that is either alive or has living descendants. 
Let the distance between $t$ and the root of the tree be $d_t$ and the distance between $a$ and the root of the tree be $d_a$.
The pruning operation, then, takes $\mathcal{O}(d_t - d_a)$ steps. 
In the worst case, this value will be equal to the number of elapsed generations.

However, for case 3 to occur, all taxa from $t$ to $a$ (including $a$ but not including $t$) must have already been removed.
Consequently, $d_t - d_a$ case 1 removal operations must happen for every case 3 removal operation.
Thus, the amortized time to remove the sequence of taxa from $t$ to $a$ (including both $t$ and $a$) is:

\[
\frac{2(d_t - d_a) + 1}{d_t - d_a + 1}
\]

Since

\[
\lim_{d_t - d_a\to\infty} \frac{2(d_t - d_a) + 1}{d_t - d_a + 1} = 2
\],

the amortized time complexity of pruning is bounded by a constant, i.e. it is $\mathcal{O}(1)$.


\end{proof}

% TODO: missing proof
% \begin{theorem}{Average Space Complexity of Perfect Tracking with Pruning}
% \label{thm:perfect-tracking-with-pruning-space}
% Average case is best described by the upper-case-lambda coalescent, but math on that is maybe
% less well understood than Kingman coalescent. Kingman coalescent makes a number of more limiting
% assumptions, but most of them are actually fairly appropriate for most phylogeny tracking situations
% (fixed-size population, wright-fisher model). The biggest problem is that it assumes only bifurcations
% (no multi-furcations). However, a strictly bifurcating tree is the worst case for average space complexity
% (a series of non-bifurcating trees is the absolute worst case, but under random selection it would
% quickly turn into a pretty good (best?) case). So, if we can't do an exact proof based on the upper
% case lambda coalescent, we can still do worst average case proof based on the Kingman coalescent
% Oh wait, maybe bifurcations aren't the worst case? Apparently there is a paradoxical result that shows
% that the Kingman Coalescent decreases fastest???

% Okay, here's the thing - all of these different models behave in ways that populations being tracked
% might behave. The argument to make here is that for a ton of them, the size of the tree (Ln) has been
% shown to be constant, proportional to n, limited by some distribution or asymptote, etc. It may even
% be sufficient to say that they come down from infinity. We end up with a space complexity something like
% O(generations + Ln) which in the worst case could translate to something like O(generations + 2*pop_size^2),
% except I think we can actually prove it's actually something more like O(generations + pop_size + log(pop_size))
% slight risk that pop_size should be multiplied by log(pop_size)

% \end{theorem}

% \begin{proof}
% \label{prf:perfect-tracking-with-pruning-space}

% \end{proof}

\subsubsection{Performance in parallel and distrbuted environments}
\label{sec:perfect-tracking-pruning-distrbuted}

% @MAM interesting thought: you can't prune all unifurcations when taking the second (distributed) approach to pruning-enabled perfect tracking
% @MAM another interesting thought: hstrat annotations can be smaller than UUID labels (128 bits)
% TODO: Matthew, double-check this @MAM this is ocmpletely reasonable
Like naive perfect tracking, pruning-enabled perfect tracking can proceed via either a fully-centralized database or through localized record-keeping.

Centralized pruning-enabled perfect tracking has all of the same issues in parallel and distributed environments that naive perfect tracking has.
It also adds an additional cost: parallel processes must be synced up before a pruning operation can be performed.
Otherwise, the algorithm risks incorrectly pruning a lineage that it will later learn produced a new artifact.
This requirement would translate to a substantial speed cost in practice.

In a localized model, when the last remnant of an immigrant lineage is eliminated an extinction notification would need to be propagated back to the processing element that lineage immigrated from.
Records for lineages that extinct locally but had generated emigrees to other processing elements would need to be retained until extinction notifications for all emigrees are confirmed.
Even for lineages that span few processing elements, repeat migrations between a single pair of processors could wind up long breadcrumb trails that necessitate extensive cleanup operations.

\subsubsection{Space complexity}
\label{sec:perfect-tracking-pruning-space}

As with naive (i.e., unpruned) perfect tracking, the primary memory cost of pruned perfect tracking comes from maintaining the tree of records, $T$.
Thus, our analysis in this section will focus on the size of $T$.
Technically, the worst-case size order of growth of perfect tracking with pruning is the same as for naive perfect tracking (see Theorem \ref{thm:perfect-tracking-space}).
However, whereas that order of growth is also the best case for naive perfect tracking, it is a somewhat pathological case under pruning.
When using pruning, such a case would only occur when every object is copied before being removed (i.e., no lineages go extinct).
There are some legitimate reasons these cases might occur, but most of them are fairly esoteric (see Section \ref{sec:perfect-tracking-space-supp}).

% These scenarios might legitimately occur if objects are never removed or the number of objects is allowed to continuously grow, but in these cases $N$ would also be growing continuously.
% They could potentially also happen when tracking digital objects that are copied with very small amounts of stochasticity.

With pruning, the asymptotic behavior of the average case turns out to be substantially better than that of the worst case. 
Luckily, in the vast majority of applications, the average case is more practically informative than the asymptotic behavior of the worst case.
Here, we investigate the expected size of $T$ (which we will call $E(|T|)$) in the average case as the population size ($N$) and number of time points ($G$) increase.
Of course, investigating the average case requires making some assumptions about the scenarios where this algorithm may be used.
An expedient assumption to make is that objects to be replicated are selected at random from the population (i.e., drift).
This process is described by a neutral model.
While this assumption is obviously not completely true in most cases, it will turn out that $E(|T|)$ is actually smaller for most realistic cases (e.g. any evolutionary scenario with directional selection).

Conveniently, the asymptotic behavior of $E(|T|)$ as $N$ increases has been studied extensively by evolutionary biologists under a variety of random selection models \citep{berestyckiRecentProgressCoalescent2009, tellierCoalescenceMultipleBranching2014, nordborgCoalescentTheory2019}.
The relevant branch of mathematics is coalescence theory, which describes the way particles governed by random processes agglomerate over time.
There are a variety of models of random selection, all of which yield different behavior.

Neutral models that could plausibly describe a process of replicating digital artifacts experience ``coalesence'' events.
In these events, the population establishes a new, more-recent common ancestor as a result of old lineages dying out.
This observation implies that when we run our perfect tracking with pruning algorithm for sufficiently long periods of time, on average we should expect our tree $T$ to have a long unifurcating ``stem.''
The most recent node in this chain will be the most recent common ancestor (MRCA) of the whole extant population.
The descendants of the MRCA will form a sub-tree, which we will call $T_c$.

The expected size of $T_c$ (denoted $E(|T_c|)$ here), then, will be the dominant factor affecting $E(|T|)$.
$E(|T_c|)$ is closely related to a value commonly studied in coalescence theory literature called the expected total branch length (usually denoted $L_n$ or $T_{tot}$; we will refer to it as $L_n$).
$L_n$ assumes that the phylogeny is represented by a tree, $T'$, that contains the leaf nodes, MRCA, and any internal nodes from which multiple lineages diverge (i.e., no unifurcations).
Edges in this tree have weights that represent the amount of time between the origin of the parent and the origin of the child.
$L_n$ is the sum of these weights.
Although the asymptotic behavior of $L_n$ depends on specifics of the evolutionary process, it scales sub-linearly with respect to population size for all realistic models that have been investigated in the literature to our knowledge \citep{gnedinLcoalescentsSurvey2014} \footnote{A possible exception is spatial models, which have recieved less formal analysis but are known to exhibit slower coalescence \citep{berestyckiRecentProgressCoalescent2009}}.  
How is this possible?
Under realistic neutral models, various subsets of the population will coalesce to common ancestors much faster than the whole population does \citep{nordborgCoalescentTheory2019}.
Consequently, there are many shallow branches.
The farther back in $T_c$ you go, the fewer branches there are.
As a result, the marginal impact of increasing $N$ diminishes as $N$ increases.

What do these observations tell us about $E(|T_c|)$?
% @ELD Figure here would deifnitely be a good idea
Because $T_c$ contains unifurcations, each edge in $T'$ corresponds to a chain of unifurcating nodes in $T_c$.
The length of this chain in $T_c$ will, on average, be proportional to the weight of the corresponding edge in $T'$.
Thus, the expected value of $L_n$ is almost identical to $E(|T_c|)$.
However, there is one important difference.
The reason $T'$ is able to scale sub-linearly with $N$ is that as $N$ goes to infinity the lengths of the branches to the leaf nodes (i.e., the extant population) approach zero \citep{nordborgCoalescentTheory2019, delmasAsymptoticResultsLength2008, drmotaAsymptoticResultsConcerning2007}.
In contrast, the space taken up by each of the $N$ leaf nodes in $T_c$ is a fixed constant and cannot drop to zero.
Thus, $E(|T_c|)$ is $\mathcal{O}(L_n + N)$.
Note that the time to coalescence depends entirely on the population size ($N$); the total number of generations elasped since the start of the process ($G$) does not affect the asymptotic growth rate of $E(|T_c|)$.

The precise asymptotic scaling of $L_n$ with respect to population size $N$ depends on the exact model of random replication used \citep{tellierCoalescenceMultipleBranching2014}.
Under the best-studied model, Kingman's Coalescent, it is $\mathcal{O}(log(N))$ \citep{kingmanCoalescent1982, delmasAsymptoticResultsLength2008}.
Under another reasonable model, the Bolthausen-Sznitman coalescent, it is $\mathcal{O}(\frac{N}{log(N)})$ \citep{drmotaAsymptoticResultsConcerning2007}.
% Some exotic models could also produce the star-shaped coalescent, in which all $N$ offspring descend directly from the most recent common ancestor. 
% While this scenario actually takes less memory in total, its asymptotic growth is $\mathcal{O}(N)$.
Because these growth rates are all less than $N$, $E(|T_c|)$ ends up being dominated by $N$ term. 
Thus the overall memory usage of pruning-enable perfect tracking is $\mathcal{O}(N + G)$.
Trivially, the ``stem'' part of $T$ can be pruned off when coalescence occurs, producing a bound of $\mathcal{O}(N)$.

%TODO: This could move to supplement if we need space
While $L_n$ is not the primary factor determining the memory use of this algorithm, it still has a substantial practical impact on resource usage.
Thus, it is worth briefly exploring our practical expectations of its size.
Earlier, we mentioned that in practice $T$ will usually be smaller than predicted by neutral models.
These models represent a scenario called ``neutral drift'' in evolutionary theory. 
Under drift, coalescence can be slow because it needs to happen entirely by chance.
When some members of the population are more likely to be selected than others and that selective advantage is hereitable (assuming population size $N$ is stable), we expect to see ``selective sweeps'' in which those members of the population outcompete others.
A selective sweep will usually speed up coalescence (unless it is competing with a different simultaneous selective sweep) \citep{neherGeneticDraftSelective2013}.
Thus, in general, in the presence of non-random selection, we expect $T_c$ to be much smaller than we would expect under drift.
Note, however, that the Bolthausen-Sznitman coalescent mentioned above does model scenarios with selection.

Would we ever expect $T_c$ to be larger than under drift?
Only if lineages coexist for longer than we would expect by chance.
Such a regime can theoretically occur when there is selection for stable coexistence or diversity maintenance.
However, such a regime also involves introducing non-random (balancing/stablizing) selection.
Thus, in practice, while these regimes maintain multiple deep branches, there is often frequent coalesence within those branches.
Theoretically, though, extreme pressure for diversity could cause $E(|T|)$ to scale faster than $\mathcal{O}(N)$.
Such scenarios could potentially be created via ecological interactions.  % TODO cite selection schemes as ecologies?

One factor that we have glossed over in this analysis is the possibility for $N$ to change over time.
For reasons that are explored further in Section \ref{sec:perfect-tracking-space-supp}, allowing population size $N$ to fluctuate complicates mathematical analysis but does not substantially alter our conclusions.

