\section{Perfect Tracking Algorithm} \label{sec:perfect-tracking}

Traditionally, the problem of recording the phylogenies of replicating digital 
entities has been solved using a ``perfect tracking'' approach where all replication 
events are recorded and stored in a tree (or forest, if there are multiple roots) [TODO: Cite phylotrackpy].
In this section, we present results on the time and space complexity of perfect tracking;
these results will serve as a basis of comparison for our curation policy algorithms in the context
of lineage tracking.

Perfect tracking algorithms are designed to plug into the ongoing replication proces.
As such, they exist in three separate steps: 1) initialization, 2) handling the birth of new objects, and 3) (optionally) handling the removal of existing objects.
The initialization step happens at the beginning of the process being measured.
The birth handling sub-algorithm gets called any time a new object is created.
Similarly, the removal handling sub-algorithm gets called any time an object is removed from the set of currently active objects.
The naive perfect tracking algorithm is fairly straightforward and is formally desribed in algorithm \ref{alg:perfect-tracking}. 
%For the purposes of analysis, we assume that this algorithm is being used in the context of a digital evolving system with a population of size $N$ that runs for $G$ generations. 

\begin{algorithm}
    \caption{Perfect phylogenetic tracking}
    \label{alg:perfect-tracking}
    \begin{algorithmic}[1]
    \STATE{Define $T:=$ an empty tree}
    \STATE{Define $P:=$ a population of size $n$}
    \STATE{Add node representing common ancestor to $T$}
    \FOR{$i = 0$ to $g$}
        \FOR{$j = 0$ to $n$}
            \STATE{Add node representing $P_j$ to $T$}
            \STATE{Add edge in $T$ from parent of $P_j$ to $P_j$}
        \ENDFOR
    \ENDFOR
    \RETURN $T$
    \end{algorithmic}
\end{algorithm}

The naive perfect tracking algorithm is very inexpensive in terms of time complexity:

\begin{theorem}{Naive Perfect Tracking Time Complexity}
\label{thm:perfect-tracking-time}
The naive perfect tracking algorithm can be implemented in constant time ($\mathcal{O}(1)$) per birth event.
\end{theorem}

\begin{proof}
\label{prf:perfect-tracking-time}
For the purposes of this proof, we use the RAM model of computation. 
Initialization takes a single simple operation and only happens once. 
Thus, it is trivially constant-time.
Handling a birth event takes two simple operations.
Thus, each run of the birth-handling algorithim trivially takes constant time.
The birth-handling algorithm will run once per birth event.
Although the number of birth events is likely not constant with respect to the size of the population or the number of generations,
these birth events are not part of the tracking algorithm and would have happened regardless of whether phylogeny tracking was in place.
Consequently, adding phylogeny tracking will only increase the time complexity of a computational process to which it is added by a constant amount per birth event.
\end{proof}

However, naive perfect tracking has a very large cost in terms of space complexity:

\begin{theorem}{Naive Perfect Tracking Space Complexity}
\label{thm:perfect-tracking-space}
The space complexity of the naive perfect tracking algorithm is $\mathcal{\theta}\sum_{t=0}^{G} N_t$, where $N_t$ is the number of new objects born at time $t$ and $G$ is the number of time steps.
\end{theorem}

\begin{proof}
\label{prf:perfect-tracking-space}
The birth-handling algorithm adds a node to the tree ($T$) each time it is called.
It is called once for every birth event.
No nodes are ever removed from $T$.
Therefore, the number of nodes in $T$ is the total number of nodes ever created.
\end{proof}
    

For most substantial applications, this space complexity is prohibitive.
Instead, most practitioners use an alternative perfect tracking algorithm that includes a ``pruning'' step to reduce memory use (see algorithm \ref{alg:perfect-tracking-pruning}).
This algorithm removes all parts of the phylogeny that are no longer relevant to explaining the history of currently extant objects.
Thus, like naive perfect tracking, it does not introduce any uncertainty into down-stream analyses.

%Note that another important axis of comparison is the amount of information retained
%In this ``perfect tracking'' approach, no information is lost
%As we will show in this section, this perfect tracking
%approach can be carried out efficiently on traditional hardware. 

% Worst-case space complexity is O(N*G), which happens when each member of the population has exactly one offspring
% Average-case is probably more informative. Due to coalesence, even "average" case (if we assume everything is random)
% will still be worse than we would usually see in practice (except when we're running controls with no selection)
%
% For average case, each member of P at time i has a ((N-1)/N)^N chance of not being picked.
% As explained here: https://math.stackexchange.com/questions/41519/expected-number-of-unique-items-when-drawing-with-replacement
% this translates to an expected value of N - ((N-1)^N)/(N^(N-1))
% i.e. each element has a (N - ((N-1)^N)/(N^(N-1))))/N probability of making it to the next generation
% Double check, but allegedly the limit of this value as N goes to infinitiy is 1 - 1/e =~ 0.6321... (Wolfram Alpha agrees)
%
% Without pruning, this still leaves us with space complexity O(N*G), because each generation we add N taxa and we never delete
% anything
%
% With pruning, on time step 1 we get to eliminate approximately 37% of the nodes from time step 0
% At every time step we eliminate 37% of the nodes from the previous time step. Sometimes this also allows
% us to eliminate nodes from even earlier time steps
%
% Expected number of nodes in each generation with exactly one offspring: (1/N) * ((N-1)/N)^(N-1).
% These nodes have a 37% chance of being eliminated in the next generation
%
% Maybe try to define this all recursively?
% P0, T=0: N nodes
% P0, T=1. .63N nodes. O1: (1/N) * ((N-1)/N)^(N-1) have one offspring, O2: (1/N)^2 * ((N-1)/N)^(N-2) have 2. OX: (1/N)^X * ((N-1)/N)^(N-X) have X
% P0, T=2. .63N - .37*O1 - ? * OX


\begin{algorithm}
    \caption{Perfect phylogenetic tracking with pruning}
    \label{alg:perfect-tracking-pruning}
    \begin{algorithmic}[1]
    \STATE{Define $T:=$ an empty tree}
    \STATE{Define $A:=$ the first member of the population (universal common ancestor)}
    \STATE{Define $P:=$ a population containing $n$ descendants of $A$}
    \STATE{Add node representing $A$ to $T$}

    \FOR{$i = 0$ to $g$}
        \FOR{$j = 0$ to $n$}
            \STATE{Add node representing $P_j$ to $T$}
            \STATE{Add edge in $T$ from parent of $P_j$ to $P_j$}
            \STATE{$P^\prime$ $\gets$ $P$}            
        \ENDFOR
        \STATE{$P$ $\gets$ new population of size $n$, descended from $P^\prime$}
        \FOR{$j = 0$ to $n$}
            \IF{$P^\prime_j$ is not the parent of any member of $P$}
                \STATE{parent $\gets$ parent of $P^\prime_j$}
                \STATE{current $\gets$ $P^\prime_j$}
                \WHILE{out degree of parent = 1}
                    \STATE{Remove current from $T$}
                    \STATE{current $\gets$ parent}
                    \STATE{parent $\gets$ parent of current}
                \ENDWHILE
            \ENDIF
        \ENDFOR        
    \ENDFOR
    \RETURN $T$
    \end{algorithmic}
\end{algorithm}
\begin{theorem}{Perfect Tracking with Pruning TODO}
\label{thm:perfect-tracking-with-pruning-todo}

\end{theorem}

\begin{proof}
\label{prf:perfect-tracking-with-pruning-todo}

\end{proof}

