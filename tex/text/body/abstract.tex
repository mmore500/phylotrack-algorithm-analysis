\begin{abstract}

% TODO we should probably use the word near optimal somewhere

We introduce the streaming curation problem, which poses the question of how to satisfactorily maintain a temporally representative collection of observations they stream in on a rolling basis.
When faced with storage space limitations, ongoing data feeds necessitate continual downsampling.
Such conditions pertain to unattended data loggers, irregularly-uplinked wireless sensor network devices, and servers under continuous application monitoring, among other scenarios; here, we specially address one application of streaming curation in particular: approximate lineage tracking over distributed collections of replicating digital entities through ``hereditary stratigraphy’’ methodology.
To meet requirements of diverse streaming curation scenarios, we present a suite of streaming curation policy algorithms that trade off along a continuum of collection size and temporal coverage.
Policies span $\mathcal{O}(n)$, $\mathcal{O}(\log n)$, and constant $\mathcal{O}(1)$ orders of growth in curated collection size.
Within each order of growth, policy algorithms differ in relative prioritization of newer versus older data.
We explore two alternatives: even density distribution across history or recency-proportional distribution with greater density at more recent time points.
Policy algorithm designs explicitly pre-determine retained collection composition at each time point, enabling key optimizations to reduce storage overhead and streamline processing of incoming observations.

Applications of streaming curation to lineage inference in hereditary stratigraphy arises from interpretation of each individual as an independent observation of lineage association, strung along in an ongoing generational stream.
In practice, this translates to annotating replicators with a heritable set of checkpoint identifiers, with new checkpoints created per generation managed through streaming curation.
Recent experimental work has proven out robust recovery of phylogenetic information through prototype instrumentation built on this principle.
Here, we formalize the theoretical foundations of hereditary stratigraphy through exposition of streaming curation algorithms and resolve a lynchpin limitation to its end-to-end application: methods for scalable, whole-tree reconstruction from checkpoint annotations.
As a point of comparison, we also provide formal analysis for existing centralized methods for direct phylogenetic tracking.
Phylogenetic analysis capabilities significantly advance distributed agent-based simulations as a tool for evolutionary research.
Such decentralized tracing could extend also to other digital artifacts that proliferate through replication, like digital media and computer viruses.
Implications of underlying streaming curation algorithms extend more broadly still, encompassing wider issues of rolling record management.
\end{abstract}
